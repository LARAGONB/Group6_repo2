\documentclass[]{article}
    \usepackage{lmodern}
    \usepackage{amssymb,amsmath}
\usepackage{ifxetex,ifluatex}
\usepackage{fixltx2e} % provides \textsubscript
\ifnum 0\ifxetex 1\fi\ifluatex 1\fi=0 % if pdftex
\usepackage[T1]{fontenc}
\usepackage[utf8]{inputenc}
  \else % if luatex or xelatex
\ifxetex
\usepackage{mathspec}
\usepackage{xltxtra,xunicode}
\else
  \usepackage{fontspec}
\fi
\defaultfontfeatures{Mapping=tex-text,Scale=MatchLowercase}
\newcommand{\euro}{€}
        \fi
% use upquote if available, for straight quotes in verbatim environments
\IfFileExists{upquote.sty}{\usepackage{upquote}}{}
% use microtype if available
\IfFileExists{microtype.sty}{%
  \usepackage{microtype}
  \UseMicrotypeSet[protrusion]{basicmath} % disable protrusion for tt fonts
}{}
  \usepackage[left=2.5in,bottom=1.25in,top=1.25in,right=1in]{geometry}
  \ifxetex
\usepackage[setpagesize=false, % page size defined by xetex
            unicode=false, % unicode breaks when used with xetex
            xetex]{hyperref}
\else
  \usepackage[unicode=true]{hyperref}
\fi
\hypersetup{breaklinks=true,
bookmarks=true,
pdfauthor={},
pdftitle={Sex and social class determined survival on the Titanic},
colorlinks=true,
citecolor=blue,
urlcolor=blue,
linkcolor=magenta,
pdfborder={0 0 0}}
\urlstyle{same}  % don't use monospace font for urls
\setlength{\parindent}{0pt}
\setlength{\parskip}{6pt plus 2pt minus 1pt}
\setlength{\emergencystretch}{3em}  % prevent overfull lines
\providecommand{\tightlist}{%
\setlength{\itemsep}{0pt}\setlength{\parskip}{0pt}}
\setcounter{secnumdepth}{0}

% Customization for cos_prereg
\usepackage{longtable,booktabs,threeparttable,tabularx}
\linespread{1.5}
\newcounter{question}
\setcounter{question}{0}

%%% Use protect on footnotes to avoid problems with footnotes in titles
\let\rmarkdownfootnote\footnote%
\def\footnote{\protect\rmarkdownfootnote}

%%% Change title format to be more compact
\usepackage{titling}

\def\changemargin#1#2{\list{}{\rightmargin#2\leftmargin#1}\item[]}
\let\endchangemargin=\endlist

% Create subtitle command for use in maketitle
\newcommand{\subtitle}[1]{
\posttitle{
\begin{center}\large#1\end{center}
}
}

\setlength{\droptitle}{-2em}
\title{Sex and social class determined survival on the Titanic}
\pretitle{\begin{changemargin}{-8pc}{0pc} \centering\large Preregistration\\ \Huge}
\posttitle{\end{changemargin}}
  \author{
          Lina Aragon-Baquero\textsuperscript{1},
          Kristen Bill\textsuperscript{2},
          Leah D'Aloisio\textsuperscript{3},
          Jack Goldman\textsuperscript{4},
          Briar Hunter\textsuperscript{5,6}          \\ \vspace{0.5cm}
              \textsuperscript{1} University of Waterloo\\
              \textsuperscript{2} Wilfrid Laurier University\\
              \textsuperscript{3} University of British Columbia
Okanagan\\
              \textsuperscript{4} University of Toronto\\
              \textsuperscript{5} Laurentian University\\
              \textsuperscript{6} Toronto Zoo      }

  \def\affdep{{"", "", "", "", ""}}%
  \def\affcity{{"", "", "", "", ""}}%
  \preauthor{\begin{changemargin}{-8pc}{0pc} \centering\large}
  \postauthor{\end{changemargin}}
\date{06. October 2021}
\predate{\begin{changemargin}{-8pc}{0pc} \centering\large\emph}
\postdate{\end{changemargin}}
\usepackage{fancyhdr}
\pagestyle{fancy}
\renewcommand{\headrulewidth}{0pt}
\lhead{}
\rhead{\large\textsc{\MakeLowercase{Survival on the Titanic}}}



% Title settings
\usepackage{titlesec}
\titleformat{\section}[display]{\bfseries\Large}{\thesection}{}{}[]
\titlespacing{\section}{0pc}{*3}{*1.5}
\titleformat{\subsection}[leftmargin]{\titlerule\bfseries\filleft}{\thesubsection}{.5em}{}
\titlespacing{\subsection}{8pc}{5ex plus .1ex minus .2ex}{1.5pc}
  

% Redefines (sub)paragraphs to behave more like sections
\ifx\paragraph\undefined\else
\let\oldparagraph\paragraph
\renewcommand{\paragraph}[1]{\oldparagraph{#1}\mbox{}}
\fi
\ifx\subparagraph\undefined\else
\let\oldsubparagraph\subparagraph
\renewcommand{\subparagraph}[1]{\oldsubparagraph{#1}\mbox{}}
\fi


\begin{document}
\maketitle
\vspace{2pc}


\newcommand\Question[2]{%
   \leavevmode\par
   \stepcounter{question}
   \noindent
   \textbf{\thequestion. #1}. #2\par}

\newcommand\Answer[1]{%
    \noindent
    \textit{Registered response}: #1\par}
    
\hypertarget{study-information}{%
\section{Study Information}\label{study-information}}

\hypertarget{title}{%
\subsection{Title}\label{title}}

Sex and social class determined survival on the Titanic

\hypertarget{description}{%
\subsection{Description}\label{description}}

It is well known that survivorship upon the sinking of the Titanic was
heavily influenced by sex and class of the individual. Previous studies
reveal the role of sex and class was a strong predictor of survival due
to policies and social/physical barriers put in place prioritizing first
class and women/children Bruno S. Frey, Savage, \& Torgler (2009) Hall
(1986). This study seeks to replicate previous findings while also
statistically testing the degree to which these variables influenced
survival. Therefore, using an alternative dataset, we will run a mixed
effects model to test the effects of class and sex on survival on the
Titanic.

\hypertarget{hypotheses}{%
\subsection{Hypotheses}\label{hypotheses}}

Similar to other studies done on Titanic survivorship data, we predict
class and sex will have a strong impact on survivorship, with
women/children and higher class passengers (i.e.~first class) being more
likely to survive than men or third class passengers.

\hypertarget{design-plan}{%
\section{Design Plan}\label{design-plan}}

\hypertarget{study-type}{%
\subsection{Study type}\label{study-type}}

This is a retrospective observational study using a dataset based on
known outcomes from the historical event of the sinking of the Titanic.

\hypertarget{blinding}{%
\subsection{Blinding}\label{blinding}}

Not applicable.

\hypertarget{study-design}{%
\subsection{Study design}\label{study-design}}

This study will utilize retrospective data from a one-time natural
event, therefore data collection is constrained by information
gathered/documented from passengers and survivors. This information was
then compiled into the dataset that will be used for this study. On
account of this being a one-time event with a defined number of people,
our design is a cross-sectional observation study.

To understand the driving factors of survival aboard the titanic, we
will use a mixed effect model approach in which sex and class will be
fixed factors and age a random factor: Survival \textasciitilde{} Class
+ Sex + (1\textbar Age).

\hypertarget{randomization}{%
\subsection{Randomization}\label{randomization}}

Not applicable.

\hypertarget{sampling-plan}{%
\section{Sampling Plan}\label{sampling-plan}}

\hypertarget{existing-data}{%
\subsection{Existing data}\label{existing-data}}

As of the date of this submission, we have accessed and analyzed some of
the data relevant to the research plan, including preliminary analysis
of variables, calculation of descriptive statistics, and observation of
data distributions.

\hypertarget{explanation-of-existing-data}{%
\subsection{Explanation of existing
data}\label{explanation-of-existing-data}}

The data we will use is a dataset retrieved from
\url{https://github.com/paulhendricks/titanic}. The data provides
information on the fate of passengers of the Titanic including variables
such as class, age, sex, ticket fare, and nationality. This dataset is
an array resulting from cross-tabulating 2201 observations. Prior to
conducting our own analyses, we will not read any papers performing
statistical analyses and thus remain unaware of any summary statistics
on the data. We have only read prior articles using a different dataset
to discuss effects of social class and sex on survival. These papers
also gathered information based on interviews of the survivors following
the event.

\hypertarget{data-collection-procedures}{%
\subsection{Data collection
procedures}\label{data-collection-procedures}}

This study will use the publicly available titanic dataset found in R
(\url{https://github.com/paulhendricks/titanic}). The principal source
for the data in this package was built on previously existing data in
which two authors re-visited the death count to best represent the
historical event in this dataset Dawson (1995) Simonoff (1997). These
contributions then led to the creation of this simulated data, which is
formatted in a machine learning context for training purposes. The study
population we will use is not reproducible, as it relied on the specific
circumstances of the Titanic sinking, which we hope never happens again.
Comparative analyses, however, can be performed on similar events such
as the sinking of the Lusitania B. S. Frey, Savage, \& Torgler (2010),
but the exact conditions are unlikely to be repeated.

\hypertarget{sample-size}{%
\subsection{Sample size}\label{sample-size}}

The sample size of our study will be based on available passenger data.
The total sample size in the dataset is 1309 subjects. Since 263
passengers are missing information about their age, we will be using a
sample size of 1046.

\hypertarget{sample-size-rationale}{%
\subsection{Sample size rationale}\label{sample-size-rationale}}

Since these data are obtained from an historical event, we are
constrained to the information made available to the public. The sample
size in this dataset is not true to the total number of passengers and
crew on board due to lack of records obtained prior to boarding the
Titanic. Furthermore, not all information is provided for each passenger
in the dataset, as certain details were only able to be obtained from
the survivors. Since we are not adding observations from this historical
event, we will not be doing a power analysis. After removing missing
variables, we will use 1046 passengers from the dataset.

\hypertarget{stopping-rule}{%
\subsection{Stopping rule}\label{stopping-rule}}

Not applicable

\hypertarget{variables}{%
\section{Variables}\label{variables}}

\hypertarget{manipulated-variables}{%
\subsection{Manipulated variables}\label{manipulated-variables}}

Not applicable.

\hypertarget{measured-variables}{%
\subsection{Measured variables}\label{measured-variables}}

\textbf{The measured variables will be:}

\begin{itemize}
\item
  Passenger class on the boat = (1 = 1st; 2 = 2nd; 3 = 3rd)
\item
  Survival = (0 = No; 1 = Yes)
\item
  Sex (those of uknown sex were added to male category) - Age
\end{itemize}

\textbf{Other variables provided in the dataset, but will not be
included in this confirmatory analysis are:}

\begin{itemize}
\item
  Number of family members = number of siblings/spouses and
  parents/children on board
\item
  Fare = Passenger Fare in British pounds
\item
  Boat number = Which lifeboat a passenger was on
\item
  Home/Destination = Destination of travels
\end{itemize}

\hypertarget{indices}{%
\subsection{Indices}\label{indices}}

To understand the summary statistics of the Titanic data, means will be
extracted from the raw data. No other indices will be used for these
data. Estimated marginal means will be used for multiple comparisons of
simple logistic regressions.

\hypertarget{analysis-plan}{%
\section{Analysis Plan}\label{analysis-plan}}

To understand how social class and gender (independent) drive
survivorship (dependent) on the Titanic, we will use general mixed
effect models (GLMMs) with age as a random effect in the model (R
package: lme4). We will test that accounting for age as a random effect
improves the model. Interactions between gender and social class will
also be explored to assess whether the interaction of the two variables
are driving the outcome. Assumptions of the GLMMs will be confirmed.

\hypertarget{statistical-models}{%
\subsection{Statistical models}\label{statistical-models}}

To understand how social class and gender drive survivorship on the
Titanic, we will use generalized mixed effect models with age as a
random effect in the model (R package: lme4). Interactions between
gender and social class will also be explored to assess whether the
interaction of the two variables are driving the outcome.
Multicollinearity will also be assessed using variable inflation factor
(VIF) (R package: car). Chi-squared test for comparisons of models will
be used to test the differences among models. Estimated marginal means
(R package: emmeans) with Bonferroni adjustments will be used for
multiple comparisons for simple logistic multiple regression models.
Model prediction assessment will be done using confusion matrices.

\hypertarget{transformations}{%
\subsection{Transformations}\label{transformations}}

No transformations will be used for these data unless transformations
are needed to meet assumptions. There are several variables that are
binary and will be coded. The dependent variable survived will be coded
as ``yes'' and ``no.'' Gender, which is a predictor variable, will be
coded as ``male'' or ``female.''

\hypertarget{inference-criteria}{%
\subsection{Inference criteria}\label{inference-criteria}}

We will use the standard p\textless.05 criteria from the Chi-squared
analysis to determine the most parsimonious glm model, which would
suggest that the models are significantly different from those expected
if the null hypothesis were correct. AIC and BIC values will also be
compared between models to further assess which model is the most
parsimonious model. The post-hoc estimated marginal means analysis,
adjusted using Bonferroni.

\hypertarget{data-exclusion}{%
\subsection{Data exclusion}\label{data-exclusion}}

Passengers who did not have their age documented in the dataset will be
excluded from the study. Outliers will be included in the analysis. No
other checks will be performed to determine the eligibility for
inclusion of data.

\hypertarget{missing-data}{%
\subsection{Missing data}\label{missing-data}}

Passengers who did not have their age in the dataset will be
automatically excluded from the analysis.

\hypertarget{exploratory-analyses-optional}{%
\subsection{Exploratory analyses
(optional)}\label{exploratory-analyses-optional}}

\href{Interactive\%20effects\%20of\%20Best\%20Titanic\%20model.png}{Interactive
effects for most parsimonious Titanic model predicting survivability
using gender and class as interactive factors.}

\href{Residual\%20plots\%20for\%20the\%20Best\%20titanic\%20model.png}{Residual
plots for the Best Titanic model. These figures visualize the residuals
to see if there are any differences in the variability of residuals as
the value for each predictor variable increases.}

\hypertarget{other}{%
\section{Other}\label{other}}

\hypertarget{other-optional}{%
\subsection{Other (Optional)}\label{other-optional}}

Enter your response here.

\hypertarget{references}{%
\section{References}\label{references}}

\hypertarget{section}{%
\subsection{}\label{section}}

\vspace{-2pc}
\setlength{\parindent}{-0.5in}
\setlength{\leftskip}{-1in}
\setlength{\parskip}{8pt}

\noindent

\hypertarget{refs}{}
\begin{CSLReferences}{1}{0}
\leavevmode\hypertarget{ref-dawson_unusual_1995}{}%
Dawson, R. J. MacG. (1995). The {``{Unusual} {Episode}''} {Data}
{Revisited}. \emph{Journal of Statistics Education}, \emph{3}(3), 7.
doi:\href{https://doi.org/10.1080/10691898.1995.11910499}{10.1080/10691898.1995.11910499}

\leavevmode\hypertarget{ref-frey_cesifo_2009}{}%
Frey, Bruno S., Savage, D. A., \& Torgler, B. (2009). {CESifo} {Working}
{Paper} no. 2551, 32.

\leavevmode\hypertarget{ref-frey_interaction_2010}{}%
Frey, B. S., Savage, D. A., \& Torgler, B. (2010). Interaction of
natural survival instincts and internalized social norms exploring the
{Titanic} and {Lusitania} disasters. \emph{Proceedings of the National
Academy of Sciences}, \emph{107}(11), 4862--4865.
doi:\href{https://doi.org/10.1073/pnas.0911303107}{10.1073/pnas.0911303107}

\leavevmode\hypertarget{ref-hall_social_1986}{}%
Hall, W. (1986). Social class and survival on the {S}.{S}. {Titanic}.
\emph{Social Science \& Medicine}, \emph{22}(6), 687--690.
doi:\href{https://doi.org/10.1016/0277-9536(86)90041-9}{10.1016/0277-9536(86)90041-9}

\leavevmode\hypertarget{ref-simonoff_unusual_1997}{}%
Simonoff, J. S. (1997). The {``{Unusual} {Episode}''} and a {Second}
{Statistics} {Course}. \emph{Journal of Statistics Education},
\emph{5}(1), 4.
doi:\href{https://doi.org/10.1080/10691898.1997.11910524}{10.1080/10691898.1997.11910524}

\end{CSLReferences}

\end{document}